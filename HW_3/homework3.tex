\documentclass{article}
\usepackage{fullpage}
\usepackage{enumitem}
\usepackage{verbatim}
\usepackage{pdflscape}
\usepackage{graphicx}
\usepackage{tikz}
\usetikzlibrary{trees}
\begin{document}
\title{CMSI 488 Homework \#3}
\author{Zane Kansil \& Edward Bramanti}
\maketitle
\begin{enumerate}
    \item Here's a grammar:
        \begin{verbatim}
            S -> A M
            M -> S?
            A -> 'a' E | 'b' A A
            E -> ('a' B | 'b' A)?
            B -> 'b' E | 'a' B B
        \end{verbatim}
        \begin{enumerate}
            \item Describe in English, the language of this grammar.
            \item Draw a parse tree for the string abaa.
            \item Prove or disprove: This grammar is LL(1).
            \item Prove or disprove: This grammar is ambiguous.
        \end{enumerate}
    \pagebreak
    \item Here's a grammar that's trying to capture the usual expressions, terms, and factors, while considering assignment to be an expression.
        \begin{verbatim}
            EXP         -> ID ":=" EXP | TERM TERM_TAIL
            TERM_TAIL   -> ("+" TERM TERM_TAIL)?
            TERM        -> FACTOR FACTOR_TAIL
            FACTOR_TAIL -> ("*" FACTOR FACTOR_TAIL)?
            FACTOR      -> "(" EXP ")" | ID
        \end{verbatim}
        \begin{enumerate}
            \item Prove that this grammar is not LL(1).
            \item Rewrite it so that it is LL(1).
        \end{enumerate}
    \pagebreak
    \item Write an attribute grammar for the grammar in the previous problem. Your attribute grammar should describe the obvious run-time semantics.
        \verbatiminput{attributegrammar.txt}
    \pagebreak
    \item Write an attribute grammar for evaluation (using the notation introduced in this class), whose underlying grammar is amenable to LL(1) parsing, for polynomials whose sole variable is x and for which all coefficients are integers, and all exponents are non-negative integers. The following strings must be accepted.
        \begin{itemize}
            \item \begin{verbatim}2x\end{verbatim}
            \item \begin{verbatim}2x^3+7x+5\end{verbatim}
            \item \begin{verbatim}3x^8-x+x^2\end{verbatim}
            \item \begin{verbatim}3x-x^3+2\end{verbatim}
            \item \begin{verbatim}-9x^5-0+4x^100\end{verbatim}
            \item \begin{verbatim}-3x^1+8x^0\end{verbatim}
        \end{itemize}
    \pagebreak
    \item Write a command-line application in Ruby, Clojure, JavaScript, or Python that evaluates polynomials from the language you defined above. The first argument should be the polynomial and the second is the value at which to evaluate the polynomial. Here are some example runs:
    \begin{verbatim}
        $ ruby evalpoly.rb "2x" 10
        20.000000
        $ ruby evalpoly.rb "2x^3+7x+5" 2
            35.000000
        $ ruby evalpoly.rb "3x^8-x+x^2" 1
            3.000000
        $ ruby evalpoly.rb "3x-x^3+2" 0
            2.000000
    \end{verbatim}
    Note that for this problem it is not necessary for you to build a seriously structured project with separate modules for scanning, parsing, error handling, abstract syntax tree construction and evaluation. Instead, make a very short and sweet script. There are no spaces in the polynomial strings so lexical analysis is no big deal; just bang out the code as a simple script. \\
    \pagebreak
\end{enumerate}
\end{document}